% $Header: /cvsroot/latex-beamer/latex-beamer/solutions/generic-talks/generic-ornate-15min-45min.en.tex,v 1.5 2007/01/28 20:48:23 tantau Exp $

%\documentclass[handout]{beamer}
%\usepackage{pgfpages}
%\pgfpagesuselayout{2 on 1}[a4paper, border shrink=5mm]

\documentclass{beamer}

\mode<presentation>
{
  \usetheme{Warsaw}
  % or ...

  \setbeamercovered{transparent}
  % or whatever (possibly just delete it)
}
\mode<handout>
{
  \usetheme{boxes}
}


\usepackage[english]{babel}
\usepackage[latin1]{inputenc}

\usepackage{times}
\usepackage[T1]{fontenc}
% Or whatever. Note that the encoding and the font should match. If T1
% does not look nice, try deleting the line with the fontenc.

\title[Linux CS Primer] {Linux for CS Students}

\subtitle {A primer}

\author{Ohad Lutzky}

\institute
{
Department of Computer Science\\
Technion IIT
}

\subject{Talks}
% This is only inserted into the PDF information catalog. Can be left
% out. 

% Delete this, if you do not want the table of contents to pop up at
% the beginning of each subsection:
\AtBeginSubsection[]
{
  \begin{frame}<beamer>{Outline}
    \tableofcontents[currentsection,currentsubsection]
  \end{frame}
}

% If you wish to uncover everything in a step-wise fashion, uncomment
% the following command: 

%\beamerdefaultoverlayspecification{<+->}

\begin{document}

\begin{frame}
  \titlepage
\end{frame}

\begin{frame}{Outline}
  \tableofcontents
  % You might wish to add the option [pausesections]
\end{frame}

% Since this a solution template for a generic talk, very little can
% be said about how it should be structured. However, the talk length
% of between 15min and 45min and the theme suggest that you stick to
% the following rules:  

% - Exactly two or three sections (other than the summary).
% - At *most* three subsections per section.
% - Talk about 30s to 2min per frame. So there should be between about
%   15 and 30 frames, all told.

\begin{frame}{A note about text editors}
  \begin{itemize}
    \item Love your text editor or hate your homework
    \item Important features to have
      \pause
      \begin{itemize}
        \item Syntax highlighting \pause
        \item Automatic indentation \pause
        \item Some form of autocompletion \pause
        \item Jump to error on compilation
      \end{itemize}
      \pause
    \item VIM does it all (\texttt{vimtutor})
      (\href{http://vim.org}{http://vim.org})
    \item Cream is an easier VIM
      (\href{http://cream.sourceforge.net}{http://cream.sourceforge.net})
      \pause
    \item So does emacs
  \end{itemize}
\end{frame}

\section{Compiling Programs}

\subsection{Using GCC Directly}

\begin{frame}{GCC}{The GNU Compiler Collection}
  \begin{itemize}
    \item Modern, full-featured compiler suite
    \item Compiles and links
    \item Supports many languages (C, C++, Pascal, Java\ldots)
    \item T2 currently has version 3.4, modern Linuxes have 4.2
    \item Available for Windows in Cygwin and DevCPP
      \pause
    \item Quite different from Borland C++ or Visual C++
  \end{itemize}
\end{frame}

\begin{frame}{Basic usage}
  \begin{itemize}
    \item Compiling and linking (for programs):\\
      \texttt{gcc -o my\_app my\_app.c} \\
    \item This
      \begin{enumerate}
        \item compiles your program into object code
        \item links your program as an executable
      \end{enumerate}
      (learn more about it in ATAM)
    \item Running your program:\\
      \texttt{\textbf{./}item}
  \end{itemize}
\end{frame}

\begin{frame}{Compiler flags}
  \begin{itemize}
    \item \texttt{-ansi} --- Conform to ANSI standard
    \item \texttt{-Wall} --- All warnings
    \item \texttt{-pedantic-errors} (no space!) --- be extra pedantic. Also,
      warnings count as errors
    \item \texttt{-g} --- debug symbols (required for using debuggers)
  \end{itemize}

  \texttt{gcc -ansi -Wall -pedantic-errors -o $\backslash$ \\
  ~~my\_prog my\_prog.c}

~

{\tiny Note: Long lines can be broken using $\backslash$}
\end{frame}

\begin{frame}{Compiling object files}
  \begin{itemize}
    \item \texttt{-c} --- no linking
    \item By \item{convention}, object files have \texttt{.o} extensions
    \item Otherwise, same flags as before
  \end{itemize}

  \texttt{gcc -ansi -Wall -pedantic-errors $\backslash$ \\
  ~~\textbf{-c} -o my\_lib.\textbf{o} my\_lib.c}
\end{frame}

\begin{frame}{Linking your application}
  \begin{itemize}
    \item Linking gives you a runnable application
    \item We use \texttt{gcc} for linking (it, in turn, runs \texttt{ld})
    \item Avoid specifying multiple \texttt{.c} files---compile objects instead
  \end{itemize}

  \texttt{gcc -o my\_app main.o lib1.o lib2.o \ldots}
\end{frame}

\begin{frame}{Common caveats}
  \begin{itemize}
    \item Circular dependencies
    \item Unit testing
    \item \ldots
  \end{itemize}
\end{frame}

\subsection{Makefiles}

\begin{frame}{Why makefiles?}
  \begin{itemize}
    \item Recompiling happens a \emph{lot}
    \item \texttt{-Wall -ansi -pedantic-errors -kimchi -\ldots}
    \item Make recompiles only what needs to be compiled
    \item Great for distributing programs
      \pause
    \item Sometimes\footnote{Operating Systems} required by course staff
  \end{itemize}
\end{frame}

\begin{frame}{In practice}
  \begin{enumerate}
    \item Create your \texttt{Makefile} (or \texttt{makefile})
    \item Run \texttt{make}
    \item Debug and fix your code
    \item Return to step 2
  \end{enumerate}
\end{frame}

\begin{frame}{A sample makefile}
\small
\texttt{\# Comments start with \# \\
\# Variable declarations look like this: \\
\# my\_var=stuff to put in the variable  \\
\# Use variables like this: \$(my\_var)  \\
CC=gcc \\
CFLAGS=-Wall -ansi -pedantic-errors \\
~\\
\# Rules look like this: \\
\# target: prerequisites \ldots \\
\# ~~~~commands \\ 
\# ~~~~\dots \\ 
my\_prog: my\_prog.c \\
~~~~\$(CC) \$(CFLAGS) -o my\_prog my\_prog.c \\
~\\
my\_lib.o: my\_lib.c \\
~~~~\$(CC) \$(CFLAGS) -c -o my\_lib my\_lib.c \\
}
\end{frame}

\begin{frame}{Running make}
  \begin{itemize}
    \item Once the \texttt{Makefile} exists, run \texttt{make}
    \item This will find the first rule and run it.
    \item Intuitively, running a rule means making sure the target is
      up-to-date with respect to the prerequisites.
    \item A rule is run like this:
      \begin{enumerate}
        \item If any of the prerequisites are targets of other rules, run
          those rules (recursively)
        \item If the target does not exist, or one of the prerequisites is
          newer than it, or there are no prerequisites, then run the commands
        \item Otherwise do nothing
      \end{enumerate}
    \item To start with a rule other than the first, run:\\
      \texttt{make my\_target}
  \end{itemize}
\end{frame}

\begin{frame}{A shortcut}{Writing makefiles doesn't have to be tedious}
\small
This makefile:

\texttt{CC=gcc \\
CFLAGS=-Wall -ansi -pedantic-errors \\
~\\
my\_prog: my\_prog.c \\
~~~~\$(CC) \$(CFLAGS) -o my\_prog my\_prog.c \\
~\\
my\_lib.o: my\_lib.c \\
~~~~\$(CC) \$(CFLAGS) -c -o my\_lib my\_lib.c \\
}

Is identical to this makefile:

\texttt{CC=gcc \\
CFLAGS=-Wall -ansi -pedantic-errors \\
~\\
my\_prog: my\_prog.c \\
my\_lib.o: my\_lib.c  \# How does make know to use -c?\\
}
\end{frame}

\begin{frame}{Some more \texttt{Makefile} tricks}
\footnotesize
\texttt{CC=gcc \\
CFLAGS=-Wall -ansi -pedantic-errors \\
OBJECTS=my\_prog.o lib1.o lib2.o \\
HEADERS=header1.h header2.h \\
\# A special 'phony' rule (there's no such file \\
\# 'all'). Compile my\_prog by default. \\
all: my\_prog \\
\# If headers change, recompile everything\\
lib1.o: lib1.c \$(HEADERS) \\
lib2.o: lib2.c \$(HEADERS) \\
\# Your main program is a library too; link it \\
\# separately. \\
my\_prog.o: my\_prog.c \$(HEADERS) \\
my\_prog: my\_prog.c \$(OBJECTS) \\
\# Another phony rule (run 'make clean') \\
clean: \\
~~~~rm -f my\_prog \$(OBJECTS)
}
\end{frame}



\section{DDD}

\begin{frame}{DDD}
  \begin{itemize}
    \item DDD is a powerful graphical layer over \texttt{gdb}.
    \item In order to use it, make sure you compile your program with
      \texttt{-g}, and run \\
      \texttt{ddd ./my\_program}
    \item The controls and commands are all identical to the ones in
      \texttt{gdb}. In fact, the bottom part of the screen is actually
      \texttt{gdb}.\footnote{The Matam tutorial slides have a reference
      for gdb}
    \item DDD's powerful graphing abilities make it handy when debugging
      usage of complex data structures.
  \end{itemize}
\end{frame}

\mode<handout> {
\begin{frame}{DDD Demo}{Empty slide for notes on the demo}
\end{frame}
}

\section*{Summary}

\begin{frame}{Summary}

  % Keep the summary *very short*.
  \begin{itemize}
  \item
    The \alert{first main message} of your talk in one or two lines.
  \item
    The \alert{second main message} of your talk in one or two lines.
  \item
    Perhaps a \alert{third message}, but not more than that.
  \end{itemize}
  
  % The following outlook is optional.
  \vskip0pt plus.5fill
  \begin{itemize}
  \item
    Outlook
    \begin{itemize}
    \item
      Something you haven't solved.
    \item
      Something else you haven't solved.
    \end{itemize}
  \end{itemize}
\end{frame}


\end{document}
